% -*- coding: UTF-8 -*-
%\documentclass[a5paper]{article}
\documentclass{article}
\usepackage{t1enc}
\usepackage{lmodern}
\usepackage[utf8]{inputenc}
\usepackage[magyar]{babel}
\usepackage{anysize}
\usepackage{cmap}
\usepackage{lmodern}

%\paperheight=21cm \paperwidth14.85cm
\marginsize{12mm}{12mm}{12mm}{12mm} % {bal}{jobb}{fels#}{als?}

\addtolength\topmargin{-\headheight} \addtolength\topmargin{-\headsep}
\addtolength\textheight{\headheight} \addtolength\textheight{\headsep}
\addtolength\textheight{\footskip}
\footskip=15pt
\ifx\pdfoutput\UnDefined \newcount\pdfoutput \fi
\ifnum0<\pdfoutput
  \pdfpagewidth\paperwidth \pdfpageheight\paperheight \pdfcompresslevel9
\else \special{papersize=14.85cm,21cm} \fi

\DeclareSymbolFont{AMSb}{U}{msb}{m}{n}%
\DeclareSymbolFontAlphabet{\mathbb}{AMSb}%
\DeclareSymbolFont{AMSa}{U}{msa}{m}{n}%
\DeclareMathSymbol{\square}       {\mathord}{AMSa}{"03}%
\DeclareMathSymbol{\blacksquare}  {\mathord}{AMSa}{"04}%
\DeclareMathSymbol{\twoheadrightarrow}  {\mathrel}{AMSa}{"10}%

\begin{document}

\begingroup\makeatletter\lowercase{\endgroup
  \def\@listi{% \begin{document} után kell!
    \leftmargin=\leftmargini%
    \topsep=5pt plus3pt minus1pt%
    %\partopsep=0pt%
    \parsep=0.5pt plus0.5pt minus0.5pt%
    \itemsep=3pt plus3pt minus1pt%
  }%
}

\noindent{\LARGE\sffamily\bfseries Relációs tervezés példák\hfill\vbox{%
  \hsize=5.5cm\scriptsize%
  \rm{}\hfill BME info99, 2001-12-07 v2\par
  {}\hfill Szabó Péter {\tt <pts@inf.bme.hu>}\par
  {}\hfill Javítások: \hfill BME info2008, 2012-06-23 v5\par
  {}\hfill Szárnyas Gábor {\tt <szarnyas@db.bme.hu>}\par
  }\par}

\smallskip
\hrule
\smallskip

A tankönyv oldalszámait $p$/$q$TK alakban jelölöm, ahol $q$ a 3. kiadásban szereplő oldalszám.

A többértékű függések témája a BSc képzés anyagában nem szerepel, az ehhez kapcsolódó feladatokat {\bf *}-gal jelöltem.

Az első néhány feladat szövege így kezdődik: Adott az $R$ relációs séma
és a rajta értelmezett funkcionális és többértékű függőségek $F$ halmaza. Kérdések:

\begin{itemize}

\item mi a függéshalmaz lezárása ($F^+$)?
\item adjuk meg a függéshalmaz egy minimális fedését!
\item mik a szuperkulcsok?
\item mik a kulcsok?
\item mik az elsődleges attribútumok?
\item mik a másodlagos attribútumok?
\item melyik (legmagasabb) normálformában van?
\item bontsuk függőségőrzően, veszteségmentesen 3NF-be (és esetleg 4NF-be)!
\item bontsuk veszteségmentesen 4NF-be!{\bf *}

\end{itemize}


\let\toot\twoheadrightarrow
\long\def\megoldas{\par\nobreak\smallskip\nobreak\noindent{\it Megoldás.}\enspace\ignorespaces}
% vvv Imp: itt is hagyhassak szóközt...
\newlength{\csillag}
\settowidth{\csillag}{{\bf *}}
\long\def\szam#1{\par\medskip\noindent{\bf #1.\hspace*{\csillag}}\quad\ignorespaces}
\long\def\szampluszfeladat#1{\par\medskip\noindent{\bf #1.*}\quad\ignorespaces}

\subsection*{Feladatok}

%\begin{enumerate}

%\item% 1
\szam{0}
$\hbox{\it RSZÓTÁR}(\hbox{\it SZERZŐ},\hbox{\it CÍM},\hbox{\it NYELV\/}^*)=
R(SCN)$. %, $SC\to N^*$.

\megoldas A függések lezárása üres.
%Hozzáadva: "A reláció"
A reláció 0NF-ben van, az egyetlen kulcs és szuperkulcs az egész $R$, minden
 attribútum elsődleges. 1NF-be bontás: $\varrho=\{SC, SN\}$. Ez egyúttal 4NF
is, mert nincs többértékű függőség és max 2 attribútum van. Függőségőrzés és
veszteségmentesség garantált.

%\item% 2
\szam{1}
$R(ABC)$, $B\to C$.

\megoldas A függések lezárása $B\to C$, a minimális fedés is ez.
$B$ minden szuperkulcsnak eleme, mert őt senki sem határozza meg. Hasonlóan
A is. Tehát a szuperkulcsok: $ABC$ és $AB$, ezek közül csak $AB$ kulcs, mert
belőle nem hagyható el attribútum, hogy még mindig szuperkulcs maradjon,
$ABC$-ből viszont elhagyható ($C)$. Az elsődleges attribútumok a kulcsok elemei,
tehát $A$ és $B$. $C$ másodlagos.

1NF-ben van (mivel minden attribútum atomi), de nem 2NF, mert van olyan
másodlagos attribútum ($C$), ami függ egy kulcs ($AB$) valódi részhalmazától
($B$-től). A $\varrho=\{AB, BC\}$ felbontás függőségőrző (mivel az összes
függőség ($B\to C$) következik a vetített függőségből ($B\to C$).
Veszteségmentes is, a 83/89TK megjegyzése alapján ($R_1:=BC$, $R_2:=AB$). $AB$
és $BC$ 4NF, mivel maximum két attribútumból állnak.

\szam{2}
$R(DOB)$, $D\to OB$, $O\to B$, $B\to O$.

\megoldas A függéshalmaz lezárása: $D\to OB$, $D\to O$, $D\to B$, $O\to B$, $B\to O$
és a triviálisak.
A minimális fedés: $D\to O$, $O\to B$, $B\to O$.
$D$ minden kulcsnak eleme, mert őt nem határozza meg senki.
Viszont $D$ már önmagában kulcs, mert ő mindenkit meghatároz. Tehát az
egyetlen kulcs $D$, a szuperkulcsok: $D$, $DO$, $DB$, $DOB$. $D$ elsődleges
attribútum (mert eleme egy kulcsnak), $B$ és $O$ viszont nem.

2NF-ben van (mert minden kulcs egyszerű), de nem 3NF, mert az $O\to B$
függésben nem igaz, hogy ($O$ szuperkulcs vagy $B$ elsődleges). A minimális
fedésből kiindulva a $\varrho_0=\{DO,DB,OB,BO,D\}$ felbontás 93/92TK alapján 3NF,
függőségőrző és veszteségmentes. $\varrho_1$-ből $D$ elhagyható, mert
részhalmaza egy másik részsémának ($DO$). Ugyanígy $BO$ is elhagyható. Tehát
a $\varrho_1=\{DO,DB,OB\}$ is veszteségmentes, függőségőrző 3NF-be bontás.
Mivel minden részséma max 2-elemű, és nincsenek többértékű függőségek,
ezért 4NF is.

Egy másik lehetséges veszteségmentes felbontás (BCNF-be, 94/93TK alapján):
az $O\to B$ (a könyvben $X\to A$)
függőség megsérti a BCNF tulajdonságot, mert $O$ nem szuperkulcs.
$R_1:=XA=OB$, $R_2:=R\setminus A=DO$. A $\varrho_2=\{OB,DO\}$ felbontásnak
minden részsémája BCNF, mert minden részséma max 2-elemű (és 4NF is, mert
nincs többértékű függőség).
Ez a felbontás veszteségmentes és függőségőrző, mert az $OB$-re vetített függőségek
($F^+$ alapján!): $O\to B$, $B\to O$ és a triviálisak; a $DO$-ra vetített függőségek:
$D\to O$ és a triviálisak; ezek uniójának lezártja ($O\to B$, $B\to O$,
$D\to O$, $D\to B$ és a triviálisak) visszaadja a teljes $F^+$-t.

94/93TK alapján máshogyan, $O\to B$-ből kiindulva is BCNF-be bonthatunk, az
előzőekhez hasonlóan belátható (szimmetriaokokból), hogy ez is
függőségőrző lesz.

\szam{3}
$S(MAJOR)$, $MR\to A$, $MAJ\to OM$, $MO\to RAJ$.

\megoldas A továbbiakban egy $H$ halmaz bővítményein azokat a halmazokat
értem, melyeknek $H$ részhalmaza. A függéshalmaz lezárása: $MO\to MAJOR$ és
halmazai (értsd: $MO$ egy bővítménye meghatározza $MAJOR$-t és részhalmazait),
$MR\to MAR$ és halmazai, $MAJ\to MAJOR$ és halmazai; és a triviálisak. Egy
minimális fedés: $MR\to A$, $MAJ\to O$, $OM\to R$, $OM\to J$.

A kulcsok: $MAJ$, $OM$, $MRJ$. Ezek tényleg szuperkulcsok és minimálisak.
Több kulcs nincs, ugyanis bármely négyelemű attribútumhalmaz tartalmazza $OM$-et,
ami kulcs, tehát ő nem lehet kulcs. Egyelemű kulcs nincs, mivel minden függőség
bal oldala legalább kételemű. Maradnak tehát a kételemű és a háromelemű
kulcsjelöltek (összesen ${5\choose 2} + {5\choose 3} = 20\,$db), ezeket kézzel ellenőrizve azt kapjuk, hogy nincs más kulcs. A
szuperkulcsok a kulcsok bővítményei.

Mindenki ($MAJOR$) elsődleges attribútum, mert mindenki eleme valamelyik
kulcsnak. Másodlagos attribútum nincs, épp ezért 3NF. Viszont nem BCNF, mert
$MR\to A$-ban $MR$ nem szuperkulcs. Az egyelemű $\varrho_1=\{MAJOR\}$
felbontás triviálisan függőségőrző, veszteségmentes 3NF-be bontás.

BCNF-be bontás (94/93TK alapján): $MR\to A$ az előzőek miatt sérti a BCNF
tulajdonságot, bontsuk tehát $\{MAR,ROJM\}$-ra. A részsémákra vetített
függőségek: $MR\to A$ és bővítményei és a triviálisak; illetve $OM\to RJ$
és bővítményei és $MRJ\to O$ és bővítményei és a triviálisak. (Rögtön
látszik, hogy a felbontás nem függőségőrző, mert újraegyesítéskor
pl.\ $MAJ\to O$ elvész.) $MAR$ már BCNF, mert $MR\to A$-ban $MR$
szuperkulcs. $ROJM$ is BCNF, mert $OM\to RJ$-ben $OM$ szuperkulcs és $MRJ\to
O$-ban $MRJ$ szuperkulcs. Mindkét részséma 4NF, mert nincs többértékű
függőség. Így tehát $\varrho_2=\{MAR,ROJM\}$ egy
veszteségmentes, nem függőségőrző, 4NF-be bontás.

\szampluszfeladat{B}
$R(\hbox{\it TANTÁRGY},\hbox{\it OKTATÓ},\hbox{\it JEGYZET\/})=R(TOJ)$, $T\toot J$.

\megoldas Mivel nincs funkcionális függőség, ezért a séma BCNF, a
függéshalmaz lezárása és minimális fedése üres, csak a teljes $TOJ$
szuperkulcs és kulcs, mindenki ($TOJ$) elsődleges.

Vajon 4NF-e a séma? 97/97TK alapján a 4NF-be tartozás feltételét sérti
$T\toot J$, mert $J$ nem üres, $J$ nem részhalmaza $T$-nek, $TJ$-n kívül
van más attribútum is ($O$), és $T$ mégsem szuperkulcs. Bontsuk fel a sémát
e sértő függőség alapján: $\varrho=\{TJ,TO\}$. Mindkét részséma 4NF, mert
max két attribútumú. A felbontás veszteségmentes, hiszen ugyancsak 97/97TK
alapján, $R_1=TJ$, $R_2=TO$ választással $(R_1\cap R_2=T)\toot
(J=R_1\setminus R_2)$. A felbontás jelenleg függőségőrző, de ez általános
esetben (csakúgy, mint BCNF-nél) nem garantálható.

\szampluszfeladat{4}
$R(\hbox{\it TANTÁRGY},\hbox{\it JEGYZET\/})=R(TJ)$, $T\toot J$.

\megoldas Mivel nincs funkcionális függőség, ezért a séma BCNF, a
függéshalmaz lezárása és minimális fedése üres, csak a teljes $TOJ$
szuperkulcs és kulcs, mindenki ($TOJ$) elsődleges. Mivel max 2 attribútum
van, ezért 4NF.

\szam{5}
$R(\hbox{\it VÁROS\/},\hbox{\it ÚT\/},\hbox{\it IR\char 95 SZÁM\/})=
R(VUI)$. $VU\to I$, $I\to V$.

\megoldas A függéshalmaz lezárása $VU\to$ bármi, $VUI\to$
bármi, $I\to V$, $UI\to V$ és a triviálisak.
%A minimális fedés $V\to I$, $U\to I$, $I\to V$.
A minimális fedés $VU\to I$, $I\to V$.
A szuperkulcsok $VUI$, $VU$, $UI$. Ezek között $VU$, $UI$ kulcs.
Minden attribútum elsődleges.
% $V$ és $U$ elsődleges attribútumok, $I$ másodlagos (mert nem eleme kulcsnak).

3NF-ben van, mert $VU\to I$-ben $VU$ szuperkulcs, $I\to V$-ben pedig $V$
elsődleges. Nincs BCNF-ben, mert $I\to V$-ben $I$ nem szuperkulcs.

Legyen $\varrho=\{R_1, R_2\}$ egy nemtriviális veszteségmentes felbontása.
Szimmetriaokokból feltehetjük, hogy $|R_1|\ge|R_2|$. A 88/87TK alján levő tétel
miatt $R_1\cap R_2\supseteq VU$ vagy $R_1\cap R_2\supseteq I$. Az előbbi nem
lehetséges (mert ellenkező esetben a nemtrivialitás miatt
$R_1\not\supseteq R_2$, amit viszont nem tudunk elkerülni), tehát
$R_1\cap R_2\supseteq I$. $|R_2|\ne1$ (mert ellenkező esetben $R_2=I$, és
emiatt $R_1\supseteq R_2$, ami ellentmond a nemtrivialitásnak).
$|R_1|\ne 3$ (mert ellenkező esetben szintén $R_1\supseteq R_2$ lenne),
tehát $R_1$ és $R_2$ is kételemű, közös elem az $I$. $R_1\cup R_2=UVI$ miatt
csak $R_1=UI$, $R_2=VI$ (vagy fordítva) lehetséges. Ez a felbontás valóban
veszteségmentes, hiszen $(R_1\cap R_2=I)\to(V=R_2\setminus R_1)$. A
részsémák BCNF-ek, mivel max 2 attribútumuk van.

Vajon az egyetlen veszteségmentes BCNF-felbontás függőségőrző-e? A vetített
függőségek lezártja $UI$-re csak a triviálisak, $VI$-re pedig $I\to V$ és a
triviálisak. Ezek uniójának lezártja $I\to V$ és a triviálisak. A $VU\to I$
függőség elveszett, tehát a felbontás nem függőségőrző. Ezzel beláttuk, hogy
a séma nem bontható fel veszteségmentesen és függőségőrzően BCNF-be (tehát
4NF-be sem).

\szam{7}
Adjunk példát olyan sémára, ami 0NF, de nem 1NF.
\megoldas A 0.\ feladat példája megfelelő. Egyébként bármely, nem atomi
attribútumot tartalmazó séma megteszi.


\szam{8}
Adjunk példát olyan sémára, ami 1NF, de nem 2NF (és nem 0NF).
\megoldas Az 1.\ feladat példája megfelelő.

Megjegyzés: a feladotot helyesebb lenne úgy kitűzni, hogy ,,adjunk példát
olyan sémára és a rajta értelmezett függőségekre\ldots''


\szam{9}
Adjunk példát olyan sémára, ami 2NF, de nem 3NF.
\megoldas A 2.\ feladat példája megfelelő.

\szam{10}
Adjunk példát olyan sémára, ami 3NF, de nem BCNF.
\megoldas A 3.\ feladat példája megfelelő.

\szam{11}
Adjunk példát olyan sémára, ami BCNF, de nem 4NF.
\megoldas A B.\ feladat példája megfelelő.

\szam{12}
Adjunk példát olyan sémára, ami 1NF.
\megoldas
Bármely, csak atomi attribútumokat tartalmazó séma megteszi, pl.\ $R(AB)$,
$A\to B$.

\szam{13}
Adjunk példát olyan sémára, ami 2NF.
\megoldas Bármely, legfeljebb 2 attribútumot tartalmazó séma, vagy csak
egyszerű kulcsokat tartalmazó séma, vagy csak elsődleges attribútumokat
tartalmazó séma megteszi, pl.\ $R(AB)$, $A\to B$.

\szam{14}
Adjunk példát olyan sémára, ami 3NF.
\megoldas Bármely, legfeljebb 2 attribútumot tartalmazó séma, vagy
baloldalon csak szuperkulcsokat tartalmazó séma, vagy csak elsődleges
attribútumokat tartalmazó séma megteszi, pl.\ $R(AB)$, $A\to B$.

\szam{15}
Adjunk példát olyan sémára, ami BCNF.
\megoldas Bármely, legfeljebb 2 attribútumot tartalmazó séma, vagy
baloldalon csak szuperkulcsokat tartalmazó séma megteszi,
pl.\ $R(AB)$, $A\to B$.

\szampluszfeladat{16}
Adjunk példát olyan sémára, ami 4NF.
\megoldas Bármely, legfeljebb 2 attribútumot tartalmazó séma, vagy
baloldalon csak szuperkulcsokat tartalmazó séma megteszi,
pl.\ $R(AB)$, $A\toot B$.


\szam{17}
Adjunk példát olyan sémára, ami 2NF, de nem 1NF.
\megoldas Ilyen séma nincsen, mert a 2NF definíciójánál kikötöttük, hogy
1NF-nek kell lennie.

\szam{18}
Adjunk példát olyan sémára, ami 3NF, de nem 2NF.
\megoldas Ilyen séma nincsen, 84/84TK alján/tetején ez be is van bizonyítva. Itt is
bebizonyítom. Tegyük fel, hogy mégis van ilyen $R$ séma. $R$ nem 2NF, tehát
(a 2NF definíciója alapján) van olyan $A$ másodlagos attribútuma, amely
függ $R$ valamely $X$ kulcsának valamely $Y$ valódi részhalmazától.
$X\to Y$ (mert $X$ kulcs).
$Y\not\to X$ (mert $Y\to X$ esetén $Y$ mindenkit meghatározna,
tehát szuperkulcs lenne, tehát tartalmazna egy $Z$ kulcsot, de ekkor $X$ nem
lenne minimális $X\supset Y\supseteq Z$ miatt).
$Y\to A$ (mert feltettük)
$A\not\in Y$ (mert $A\in Y\subset X$ esetén $A\in X$, $A$ elsődleges lenne).
A fentiek miatt
viszont $A$ és $X$ létezése ellentmond a 3NF első definíciójának
(83/82TK).\hfill$\square$

\szam{19}
Adjunk példát olyan sémára, ami BCNF, de nem 3NF.
\megoldas Ilyen séma nincsen. Bizonyítás. Egy BCNF séma minden nemtriviális
$X\to A$ függőségére
igaz (a BCNF második definíciója, 85/84TK miatt), hogy $X$ szuperkulcs. Ez
viszont azt jelenti, hogy az $X\to A$ függőség kielégíti a 3NF második
definíciójának feltételét (nevezetesen: $X$ szuperkulcs vagy $A$ másodlagos).
Tehát a séma 3NF is.$\hfill\square$

Megjegyzés: az első definíciókból is azonnal látszik.

\szampluszfeladat{20}
Adjunk példát olyan sémára, ami 4NF, de nem BCNF.
\megoldas Ilyen séma nincsen. Bizonyításvázlat: a két szuperkulcsos
definíciót kell figyelembe venni (85/84TK és 97/97TK). A BCNF definícióját
értelemszerűen terjesszük ki $X\to A$-ról $X\to Y$-ra.
Ha $|\Omega|>|XY|$ a BCNF-re nem teljesül, akkor $X$ máris szuperkulcs,
készen vagyunk. Vegyük figyelembe, hogy $X\to Y$-ból következik, hogy
$X\toot Y$ (például úgy, hogy a többértékű függőség 97/96TK definíciójában
$t_4:=t_1$, $t_3:=t_2$-t választunk).


\szam{21}
Adjunk példát olyan sémára, amely nem 2NF, de felbontható veszteségmentesen
és függőségőrzően 2NF-be.
\megoldas Az 1.\ feladat példája megfelelő. Ott egy felbontást is megadtunk
(4NF-be, amiből már következik, hogy 2NF).

\szam{22}
Adjunk példát olyan sémára, amely nem 3NF, de felbontható veszteségmentesen
és függőségőrzően 3NF-be.
\megoldas Az 1.\ és a 2.\ feladatok példái megfelelők. Ott a felbontást is
megadtunk (4NF-be is, amiből már következik, hogy 3NF).

\szam{23}
Adjunk példát olyan sémára, amely nem BCNF, de felbontható veszteségmentesen
és függőségőrzően BCNF-be.
\megoldas Az 1.\ és a 2.\ feladatok példái megfelelők. Ott a felbontást is
megadtunk (4NF-be is, amiből már következik, hogy BCNF).

\szampluszfeladat{24}
Adjunk példát olyan sémára, amely nem 4NF, de felbontható veszteségmentesen
és függőségőrzően 4NF-be.
\megoldas Az 1.\ és a 2.\ feladatok példái megfelelők. Ott a felbontást is
megadtunk.


\szam{25}
Adjunk példát olyan sémára, amely nem 2NF, és nem is bontható fel
veszteségmentesen és függőségőrzően 2NF-be.
\megoldas Nincs ilyen séma, a 92/92TK alján/tetején levő tétel miatt mindenkinek
létezik veszteségmentes és függőségőrző felbontása 3NF-be, tehát 2NF-be is.

\szam{26}
Adjunk példát olyan sémára, amely nem 3NF, és nem is bontható fel
veszteségmentesen és függőségőrzően 3NF-be.
\megoldas Nincs ilyen séma, a 92/92TK alján/tetején levő tétel miatt mindenkinek
létezik veszteségmentes és függőségőrző felbontása 3NF-be.

\szam{27}
Adjunk példát olyan sémára, amely nem BCNF, és nem is bontható fel
veszteségmentesen és függőségőrzően BCNF-be.
\megoldas Az 5.\ feladat példája megfelelő. Ott az állítást be is
bizonyítottuk.

\szampluszfeladat{28}
Adjunk példát olyan sémára, amely nem 4NF, és nem is bontható fel
veszteségmentesen és függőségőrzően 4NF-be.
\megoldas Az 5.\ feladat példája megfelelő. Ott az állítást be is
bizonyítottuk (BCNF-re, de ha valaki nem BCNF, akkor 4NF se lehet).

\szampluszfeladat{29}
Adjunk példát olyan sémára, amely nem bontható fel veszteségmentesen 4NF-be.
\megoldas Ilyen séma nincsen, lásd a 98/97TK tetején/közepén levő megjegyzést. Annak
bizonyítása, hogy mindenki felbontható, hasonlóan megy mint BCNF-re:
rekurzívan mindig két részre vágjuk, lásd még 97/97TK.

%\end{enumerate}

\end{document}
